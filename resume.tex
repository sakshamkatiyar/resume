%%%%%%%%%%%%%%%%%%%%%%%%%%%%%%%%%%%%%%%
% Deedy - One Page Two Column Resume
% LaTeX Template
% Version 1.1 (30/4/2014)
%
% Original author:
% Debarghya Das (http://debarghyadas.com)
%
% Original repository:
% https://github.com/deedydas/Deedy-Resume
%
% IMPORTANT: THIS TEMPLATE NEEDS TO BE COMPILED WITH XeLaTeX
%
% This template uses several fonts not included with Windows/Linux by
% default. If you get compilation errors saying a font is missing, find the line
% on which the font is used and either change it to a font included with your
% operating system or comment the line out to use the default font.
% 
%%%%%%%%%%%%%%%%%%%%%%%%%%%%%%%%%%%%%%
% 
% TODO:
% 1. Integrate biber/bibtex for article citation under publications.
% 2. Figure out a smoother way for the document to flow onto the next page.
% 3. Add styling information for a "Projects/Hacks" section.
% 4. Add location/address information
% 5. Merge OpenFont and MacFonts as a single sty with options.
% 
%%%%%%%%%%%%%%%%%%%%%%%%%%%%%%%%%%%%%%
%
% CHANGELOG:
% v1.1:
% 1. Fixed several compilation bugs with \renewcommand
% 2. Got Open-source fonts (Windows/Linux support)
% 3. Added Last Updated
% 4. Move Title styling into .sty
% 5. Commented .sty file.
%
%%%%%%%%%%%%%%%%%%%%%%%%%%%%%%%%%%%%%%%
%
% Known Issues:
% 1. Overflows onto second page if any column's contents are more than the
% vertical limit
% 2. Hacky space on the first bullet point on the second column.
%
%%%%%%%%%%%%%%%%%%%%%%%%%%%%%%%%%%%%%%

\documentclass[]{deedy-resume-openfont}


\begin{document}

%%%%%%%%%%%%%%%%%%%%%%%%%%%%%%%%%%%%%%
%
%     LAST UPDATED DATE
%
%%%%%%%%%%%%%%%%%%%%%%%%%%%%%%%%%%%%%%
\lastupdated

%%%%%%%%%%%%%%%%%%%%%%%%%%%%%%%%%%%%%%
%
%     TITLE NAME
%
%%%%%%%%%%%%%%%%%%%%%%%%%%%%%%%%%%%%%%


\namesection{}{Saksham Katiyar}{ \\
\href{mailto:saksham0katiyar@gmail.com}{saksham0katiyar@gmail.com} | +91 8400 6780 75
}

%%%%%%%%%%%%%%%%%%%%%%%%%%%%%%%%%%%%%%
%
%     COLUMN ONE
%
%%%%%%%%%%%%%%%%%%%%%%%%%%%%%%%%%%%%%%

\begin{minipage}[t]{0.33\textwidth} 

%%%%%%%%%%%%%%%%%%%%%%%%%%%%%%%%%%%%%%
%     EDUCATION
%%%%%%%%%%%%%%%%%%%%%%%%%%%%%%%%%%%%%%

\section{Education} 

\subsection{JSSATE, Noida}
\descript{B.Tech in Electronics and Communication Engineering}
\location{2015 - 2019 | AKTU \\ Agg. Percent: 66.28\%}
\sectionsep

\subsection{VSEC, Kanpur}
\descript{Intermediate 10+2}
\location{2013 - 2015 | ISC \\ Agg. Percent: 93.75\%}
\sectionsep

\descript{High School 10}
\location{2011 - 2013 | ICSE \\ Agg. Percent: 89.80\%}
\sectionsep

%%%%%%%%%%%%%%%%%%%%%%%%%%%%%%%%%%%%%%
%     LINKS
%%%%%%%%%%%%%%%%%%%%%%%%%%%%%%%%%%%%%%

\section{Links} 
Github:// \href{https://github.com/sakshamkatiyar}{\custombold{sakshamkatiyar}} \\
LinkedIn://  \href{https://www.linkedin.com/in/sakshamkatiyar}{\custombold{sakshamkatiyar}} \\
%YouTube://  \href{https://www.youtube.com/user/DeedyDash007}{\custombold{DeedyDash007}} \\
%Twitter://  \href{https://twitter.com/katiyarsaksham}{\custombold{@katiyarsaksham}} \\
%Quora://  \href{https://www.quora.com/Debarghya-Das}{\custombold{Debarghya-Das}}
\sectionsep

%%%%%%%%%%%%%%%%%%%%%%%%%%%%%%%%%%%%%%
%     COURSEWORK
%%%%%%%%%%%%%%%%%%%%%%%%%%%%%%%%%%%%%%

\section{Coursework}

Data Structure \& Algorithms \\
Control Systems \\
%Signals and System \\
Digital Signal Processing \\
Advanced Electronics System \\
Microprocessor + Practicum \\
Digital Design using Verilog + Practicum \\
VLSI Design + Practicum \\
\sectionsep

%%%%%%%%%%%%%%%%%%%%%%%%%%%%%%%%%%%%%%
%     SKILLS
%%%%%%%%%%%%%%%%%%%%%%%%%%%%%%%%%%%%%%

\section{Skills}
\subsection{Programming}
Python \textbullet{} C \textbullet{} Embedded C \textbullet{} Java \\
Verilog \textbullet{} VHDL \textbullet{} \LaTeX\ \textbullet{} Assembly
\sectionsep

\subsection{Tools}
ROS \textbullet{} OpenCV \textbullet{} AWS \textbullet{} MATLAB \\
XiLinx \textbullet{} VirtualBox \textbullet{} fritzing \textbullet{} Git \\
\sectionsep


\subsection{Hardware}
Arduino \textbullet{} Raspberry Pi \textbullet{} ESP 8266 \\
MSP430 \textbullet{} CC3200 \textbullet{} 8085 \textbullet{} 8051
\sectionsep

%%%%%%%%%%%%%%%%%%%%%%%%%%%%%%%%%%%%%%
%     OTHER DETAILS
%%%%%%%%%%%%%%%%%%%%%%%%%%%%%%%%%%%%%%

\section{Other Details}
\subsection{Responsibilities}
Technical Head Quanta, ECE Dept. \\
Lab Coordinator, Embedded Systems \& Robotics Lab
\sectionsep

\subsection{Languages}
English - Proficient \\
Hindi - Native \\
Spanish - Beginner
\sectionsep

%%%%%%%%%%%%%%%%%%%%%%%%%%%%%%%%%%%%%%
%
%     COLUMN TWO
%
%%%%%%%%%%%%%%%%%%%%%%%%%%%%%%%%%%%%%%

\end{minipage} 
\hfill
\begin{minipage}[t]{0.66\textwidth} 

%%%%%%%%%%%%%%%%%%%%%%%%%%%%%%%%%%%%%%
%     EXPERIENCE
%%%%%%%%%%%%%%%%%%%%%%%%%%%%%%%%%%%%%%

\section{Experience}

\href{http://tri3d.in/}{\runsubsection{Least Count}
\descript{| Computer Vision Intern}
\location{June 2018 – July 2018 | IIT-M Research Park, Chennai}
\vspace{\topsep} % Hacky fix for awkward extra vertical space
\begin{tightemize}
\item Work related to face recognition and manipulation.
\item Creating database on AWS, then implementing machine learning models to detect various features of face and quantifying them.
\item All code was reviewed, perfected, and pushed to production.
\end{tightemize}
\sectionsep}

\href{http://shellios.com/}{\runsubsection{Shellios Technolabs}
\descript{| Product Development Intern }
\location{Feb 2018 – March 2018 | JSS Step, Noida}
\begin{tightemize}
\item Deploying Cloud Services and managing distributed database on AWS.
\item Created a backbone-like framework for the cloud data storage directly from ESP32.
\end{tightemize}
\sectionsep}

\runsubsection{MNNIT}
\descript{| VLSI Design and Embedded Systems Trainee}
\location{June 2017 – July 2017 | MNNIT, Allahabad}
\begin{tightemize}
\item Synthesis and simulation of circuit designs on Xilinx ISE using Verilog. Worked on Mentor Graphics to design the layout of IC and implementation on FPGA Kit.
\item Learned the basic concepts of embedded systems and to program in assembly language on 8051 microcontroller and then using Embedded C programming.
\item Beside the mini projects, the major projects were realization of Wallace Tree Multiplier and Light to frequency converter on trainer kit.
\end{tightemize}
\sectionsep

%%%%%%%%%%%%%%%%%%%%%%%%%%%%%%%%%%%%%%
%     PROJECTS
%%%%%%%%%%%%%%%%%%%%%%%%%%%%%%%%%%%%%%

\section{Projects}
\runsubsection{e-Toll System}
\descript{| Raspberry Pi, OpenCV, Python}
\location{May 2018 – June 2018 | Smart India Hackathon, Finalist}
An advanced toll collection system based on IoT, where one RPi was used as a server maintaining database while other as a client. Number Plate was detected and matched to the database and toll was collected or exception was suitably handled.
\sectionsep

\href{https://github.com/sakshamkatiyar/chaser_drone}{\runsubsection{Drone Localization and Navigation}
\descript{| ROS, Python}
\location{October 2017 – March 2018 | e-Yantra IIT-B, Finalist}
A drone based project that involves automatic stabilization and localization of a quadcopter. It was developed using Python and ROS. Gazebo simulator was used prior to implementation
\sectionsep}

\href{https://github.com/sakshamkatiyar/cross_a_crater}{\runsubsection{Crater and Obstacle avoiding Bot}
\descript{| OpenCV, Python}
\location{October 2016 – March 2017 | e-Yantra IIT-B, Semi-Finalist}
This project shows how advance concepts of image processing can be used to process the image through a mounted camera and solve the problems related to traversing on unknown terrain with craters and other rocks while reaching the desired place.
\sectionsep}

\begin{tabular}{ll}
    \textbullet{} Optical Character Recognition & ML \\
    \textbullet{} Hand Gesture controlled Bot & Arduino \\
    \textbullet{} Maze Solving & Computer Vision
\end{tabular}

%\textbullet{} Optical Character Recognition | ML
%\textbullet{} Hand Gesture controlled Bot | Arduino
%\textbullet{} Maze Solving | Computer Vision
\sectionsep

%%%%%%%%%%%%%%%%%%%%%%%%%%%%%%%%%%%%%%
%     ACHIEVEMENTS
%%%%%%%%%%%%%%%%%%%%%%%%%%%%%%%%%%%%%%

\section{Achievements} 
\begin{tabular}{rrll}
2018 & National & Finalist & Kronothon 2.0 \\
2017 & National & 4\textsuperscript{th}/202  & Team Leader e-Yantra, IIT-B under MHRD \\
2017 & College & Hosted & Embedded Systems Workshop \\
2016 & National & Semi-Finalist & Team Leader e-Yantra, IIT-B under MHRD \\
%2014 & National & 1\textsuperscript{st}/50  & Microsoft Coding Competition, Cornell \\
2016 & International & Volunteer & International Cultural Team, WCF \\
%2014 & North Region & Semi-Finalist & Robo-Wars, COMFEST \\
%2014 & North Region & 3\textsuperscript{rd}/52 & COMFEST \\
%2013 & National & Finalist & UCMAS \\
%2012 & State & Runner Up & UCMAS \\
\end{tabular}
\sectionsep

\end{minipage} 
\end{document}  \documentclass[]{article}